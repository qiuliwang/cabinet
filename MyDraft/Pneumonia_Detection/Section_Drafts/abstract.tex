\emph{Objective:} Pneumonia detection is one of the most crucial steps in the pneumonia diagnosing system. Clinical information of patients plays an essential role in the detection of pneumonia, but few models consider this information. In this paper, a Multimodal Pneumonia Detection Network (MPDNet) is described for clinical pneumonia detection.
\emph{Method:} MPDNet simulates clinical pneumonia detection process and considers multimodal data: three-channel CT images,  chief complaints, patient age, and gender. 
(a) MPDNet firstly extracts visual features from three-channel (Lung Window, High Attenuation, Low Attenuation) images, which are transformed from one-channel grey level CT images. Different channels can provide supplementary features to each other and give qualitative information for pneumonia detection. A dedicated Recurrent CNN (RCNN) is used to extract 3D visual features from three-channel images and reduce the need for calculation resources.
(b) Then MPDNet extracts information about lesion location, symptoms, or how long patients have been ill from chief complaints, which enhances visual features extracted from CT images. A Long Short Term Memory(LSTM) network is used to analyze the semantic features of patient chief complaints. 
(c) Moreover, MPDNet uses priori information provided by age and gender to improve the decision-making process.
In MPDNet, we incorporate CT visual features, complaint semantic features, patient age, and gender to calculate joint distribution and simulates clinical pneumonia detection.
\emph{Results:} The proposed MPDNet has been extensively validated in 1002 clinical cases from The First Affiliated Hospital of Army Medical University. Our network achieves 0.945 in accuracy and has a very balanced performance in sensitivity and specificity. As far as we know, we are the first to detect pneumonic cases using large scale clinical multimodal data.
\emph{Conclusion:} Our method demonstrates that multimodal data provides more abundant information than image data only and gets very convincing results.
\emph{Significance:} While MPDNet is tailored for pneumonia detection, it can be extended and include more multimodal clinical data to give out more reliable and explainable detection results.